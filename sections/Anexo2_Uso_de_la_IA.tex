\chapter*{Anexo 2: Uso de la IA}
\addcontentsline{toc}{chapter}{Anexo 2: Uso de la IA}
\markboth{Anexo 2: Uso de la IA}{}
\label{anexo:uso_IA}
\setcounter{chapter}{1}
\renewcommand{\thechapter}{A}


En este anexo se describe el uso de herramientas de \textbf{Inteligencia Artificial} durante el desarrollo del proyecto. Para cada herramienta utilizada se detalla el propósito de su uso, así como los prompts empleados, con el objetivo de aportar transparencia y trazabilidad al proceso.

\section*{NotebookLM}

La herramienta NotebookLM se ha utilizado principalmente para la \textbf{generación de infografías} correspondientes a las fases 1, 2 y 3 del proyecto (Empatía, Definir e Idear). En cada caso, se empleó como fuente la memoria del proyecto y los resultados obtenidos tras la aplicación de las distintas técnicas.

Por ejemplo, para la generación de la infografía de la fase de Empatía, se utilizó el siguiente prompt:

\begin{quote}
«Genera una infografía del problema de acuerdo al problema a tratar en la fuente, utilizando la técnica Infografía de acuerdo a la etapa de empatía de Design Thinking. Para el contenido, debe tomarse como referencia la información de la memoria del proyecto recogida en las técnicas de los 5 porqués y conceptos del mapa mental.»
\end{quote}

Asimismo, NotebookLM se utilizó para la \textbf{generación de storyboards} a partir de las historias de usuario definidas en la fase de Prototipar. El prompt empleado para dicha tarea fue el siguiente:

\begin{quote}
«Genérame un storyboard, es una técnica visual utilizada para narrar la experiencia del usuario a través de una secuencia de imágenes o ilustraciones. Funciona como un guión gráfico que muestra cómo una persona interactuaría con un producto o servicio en diferentes situaciones. Esta herramienta ayuda a empatizar con los usuarios, detectar posibles fricciones en la experiencia y mejorar la solución antes de su desarrollo. Básate en lo que he estado comentando en la última etapa de Design Thinking, prototipado, según lo estipulado en las historias de usuario de las fuentes.»
\end{quote}

\section*{Gemini}

La herramienta Gemini se utilizó también para la \textbf{generación de storyboards} en la fase de Prototipar, a partir de las historias de usuario previamente definidas. Su uso permitió obtener representaciones visuales alternativas del flujo de interacción del usuario con la solución propuesta.

El prompt utilizado fue el siguiente:

\begin{quote}
«Genérame un storyboard, es una técnica visual utilizada para narrar la experiencia del usuario a través de una secuencia de imágenes o ilustraciones. Funciona como un guión gráfico que muestra cómo una persona interactuaría con un producto o servicio en diferentes situaciones. Esta herramienta ayuda a empatizar con los usuarios, detectar posibles fricciones en la experiencia y mejorar la solución antes de su desarrollo. Básate en lo que he estado comentando en la última etapa de Design Thinking, prototipado, según lo estipulado en las historias de usuario de las fuentes.»
\end{quote}

Además, Gemini se utilizó conjuntamente con la herramienta \textbf{Antigravity} para la realización de \textbf{juegos de roles} durante la fase de \textbf{Testeo}. Esta dinámica permitió simular la interacción entre distintos perfiles de usuario y el sistema propuesto, facilitando la identificación de posibles problemas de uso, inconsistencias en el flujo de la solución y oportunidades de mejora antes de su validación final.
