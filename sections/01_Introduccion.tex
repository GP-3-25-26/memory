\chapter{Introducción}\label{cap:introduccion}

En este primer capítulo se presenta una visión general del contexto y de la metodología empleada en el proyecto. A lo largo de la memoria se abordarán los siguientes puntos:

\begin{itemize}
    \item \textbf{¿Qué es Design Thinking?} Se introduce brevemente la metodología y las fases que la componen, que estructuran el desarrollo del trabajo.
    \item \textbf{Problema planteado.} Se describe el reto inicial propuesto y las necesidades detectadas en torno al proceso actual de análisis genómico y apoyo al diagnóstico.
    \item \textbf{Contenido de la memoria.} Se resume la organización del documento y qué se presenta en cada apartado del índice.
\end{itemize}

\section{¿Qué es Design Thinking?}

\textbf{Design Thinking} es una metodología orientada a la resolución de problemas complejos que pone al usuario en el centro del proceso de diseño. Su objetivo principal es comprender en profundidad las necesidades reales de las personas para definir correctamente el problema y generar soluciones innovadoras, viables y alineadas con el contexto en el que se aplican.

Esta metodología se caracteriza por ser iterativa, flexible y colaborativa, fomentando la experimentación continua y la validación temprana de ideas. En lugar de asumir el problema desde el inicio, Design Thinking propone explorarlo progresivamente, permitiendo refinar tanto el planteamiento del reto como la solución a medida que se avanza.

El proceso de Design Thinking se estructura en cinco fases principales, que no siempre se desarrollan de forma estrictamente secuencial, sino que pueden solaparse o repetirse según las necesidades del proyecto:

\begin{itemize}
    \item \textbf{Empatía:} fase centrada en comprender el contexto, las necesidades, motivaciones y limitaciones de los usuarios implicados.
    \item \textbf{Definir:} etapa en la que se sintetiza la información obtenida y se concreta el problema o reto a resolver de forma clara y enfocada.
    \item \textbf{Idear:} fase dedicada a la generación de múltiples ideas y posibles soluciones, fomentando la creatividad y la exploración de alternativas.
    \item \textbf{Prototipar:} consiste en materializar las ideas en prototipos tangibles, ya sean de baja o alta fidelidad, para poder evaluarlas.
    \item \textbf{Testear:} etapa en la que los prototipos se prueban y se obtiene retroalimentación, permitiendo detectar mejoras y realizar iteraciones.
\end{itemize}

En este proyecto, Design Thinking se utiliza como marco metodológico para analizar un proceso real y proponer una solución tecnológica. A lo largo de la memoria se documenta la aplicación de cada una de estas fases, asegurando la trazabilidad de las decisiones tomadas y de los resultados obtenidos en cada etapa del proceso.

\section{Problema planteado}

El reto propuesto se enmarca en el ámbito del \textbf{apoyo al diagnóstico y la predicción de enfermedades} a partir del análisis de \textbf{datos genómicos}. En el escenario actual, los genetistas reciben la información genética de los pacientes en forma de archivos \texttt{FASTQ}, que contienen la secuencia de nucleótidos obtenida tras el proceso de secuenciación.

Para llevar a cabo el análisis, es necesario \textbf{comparar manualmente} la secuencia genética del paciente con una secuencia de referencia (paciente perfecto) con el objetivo de identificar diferencias, conocidas como \textbf{variantes genómicas}. Una vez detectadas dichas variantes, el profesional debe consultar \textbf{repositorios y bases de datos médicas externas} para interpretar su relevancia clínica, determinando si son benignas o patogénicas y qué enfermedades podrían estar asociadas.

Este proceso presenta una serie de \textbf{limitaciones y problemas}:

\begin{itemize}
    \item \textbf{Alta carga manual:} gran parte del análisis se realiza de forma manual, lo que incrementa el tiempo necesario para obtener resultados.
    \item \textbf{Probabilidad de error humano:} la naturaleza manual del proceso lo hace susceptible a errores, especialmente en tareas repetitivas o complejas.
    \item \textbf{Dependencia de personal altamente especializado:} el análisis requiere genetistas con formación específica, un recurso limitado.
    \item \textbf{Falta de integración de información clínica:} el historial clínico y los antecedentes familiares del paciente no siempre se encuentran disponibles o integrados en el proceso de análisis.
\end{itemize}

Como consecuencia, se identifica la necesidad de \textbf{mejorar y apoyar el proceso actual}, facilitando el análisis genómico, reduciendo la carga manual y el margen de error, e integrando información que ayude a \textbf{orientar la toma de decisiones}.

\section{Contenido de la memoria}

El resto del documento se organiza siguiendo las fases de la metodología Design Thinking. En cada una de estas fases se presentan, por un lado, las \textbf{técnicas seleccionadas} (o su adaptación cuando ha sido necesario) junto con la justificación de su uso y, por otro lado, los \textbf{resultados obtenidos} tras la aplicación de dichas técnicas al problema planteado.

La estructura de la memoria es la siguiente:

\begin{itemize}
    \item \textbf{Fase de Empatía.} Se describen las técnicas empleadas para comprender el contexto y las necesidades de los usuarios, como el mapa mental, los 5 por qués y la infografía.
    \item \textbf{Fase de Definir.} Se presenta la síntesis de la información recopilada, la identificación de prioridades y la formulación del punto de vista del problema, haciendo uso de técnicas como la matriz 2x2, la escalera por qué - cómo, el marco de punto de vista y la infografía.
    \item \textbf{Fase de Idear.} Se detalla el proceso de generación y refinamiento de propuestas de solución mediante técnicas como el brainwriting, los mapas mentales, SCAMPER y la infografía.
    \item \textbf{Fase de Prototipar.} Se describe el diseño del prototipo y su representación mediante herramientas como el canvas, el prototipo y el storyboard.
    \item \textbf{Fase de Testear.} Se expone el proceso de validación del prototipo y la retroalimentación obtenida a partir de su evaluación.
    \item \textbf{Conclusiones y anexo.} Se recogen las conclusiones finales del proyecto y se añade un anexo en el que se documenta el uso de herramientas de inteligencia artificial durante su desarrollo.
\end{itemize}


