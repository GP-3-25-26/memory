\chapter*{Resumen}

Este trabajo presenta el desarrollo de una solución orientada al apoyo en el diagnóstico de enfermedades genéticas, aplicando la metodología \textit{Design Thinking} de forma estructurada a lo largo de sus cinco fases: Empatía, Definir, Idear, Prototipar y Testear. El objetivo principal del proyecto ha sido abordar uno de los principales retos a los que se enfrentan los genetistas en la práctica clínica: la detección y análisis manual de variantes genómicas a partir de archivos \texttt{FASTQ}, un proceso lento, complejo y propenso a errores humanos.

Durante la fase de Empatía se analizó el contexto del problema y se identificaron las dificultades existentes en los procesos actuales, apoyándose en técnicas como el mapa mental, los 5 porqués y la infografía. 

En la fase de Definir se sintetizó la información recopilada, delimitando el problema y formulando un punto de vista claro mediante técnicas como la matriz 2x2, la escalera Por qué-Cómo y el marco de punto de vista.

En la fase de Idear se generaron y refinaron múltiples propuestas de solución a través de brainwriting, mapas mentales y la técnica SCAMPER, seleccionándose finalmente una solución basada en la automatización del análisis genómico, la integración del historial clínico y el uso de modelos predictivos con explicabilidad. 

Posteriormente, en la fase de Prototipar, se desarrolló un prototipo funcional de bajo nivel utilizando React, complementado con canvas y storyboards para representar tanto la solución como los procesos no prototipados.

Finalmente, en la fase de Testear, el prototipo fue validado mediante usuarios mock y ejercicios de juego de roles apoyados por herramientas de inteligencia artificial. El testeo permitió confirmar que el flujo de trabajo resulta lógico e intuitivo y que la solución aporta un ahorro significativo de tiempo en el proceso diagnóstico. Asimismo, se identificaron áreas de mejora relacionadas con la visibilidad del contexto clínico, la explicabilidad de las predicciones y la gestión del feedback, lo que evidencia la necesidad de iterar nuevamente sobre el prototipo antes de una posible implementación real.

En conjunto, el proyecto demuestra que la solución propuesta cubre los retos principales detectados y presenta un alto potencial como producto, al tiempo que pone de manifiesto la utilidad del enfoque de Design Thinking para desarrollar soluciones complejas en el ámbito de la salud.

\vspace{.5cm}

\textbf{Palabras clave:} Design Thinking, diagnóstico genético, variantes genómicas, \texttt{FASTQ}, prototipado, inteligencia artificial, experiencia de usuario
