\chapter*{Abstract}

This work presents the development of a solution aimed at supporting the diagnosis of genetic diseases, applying the \textit{Design Thinking} methodology in a structured manner throughout its five phases: Empathize, Define, Ideate, Prototype, and Test. The main objective of the project has been to address one of the key challenges faced by geneticists in clinical practice: the manual detection and analysis of genomic variants from \texttt{FASTQ} files, a process that is slow, complex, and prone to human error.

During the Empathize phase, the context of the problem was analyzed and the difficulties present in current processes were identified, supported by techniques such as mind mapping, the Five Whys, and infographics.

In the Define phase, the collected information was synthesized, the problem was delimited, and a clear point of view was formulated using techniques such as the 2x2 matrix, the Why--How ladder, and the point-of-view framework.

In the Ideate phase, multiple solution proposals were generated and refined through brainwriting, mind maps, and the SCAMPER technique. Ultimately, a solution based on the automation of genomic analysis, the integration of clinical history, and the use of predictive models with explainability was selected.

Subsequently, in the Prototype phase, a low-fidelity functional prototype was developed using React, complemented by a canvas and storyboards to represent both the solution and the non-prototyped processes.

Finally, in the Test phase, the prototype was validated through mock users and role-playing exercises supported by artificial intelligence tools. The testing process confirmed that the workflow is logical and intuitive and that the solution provides significant time savings in the diagnostic process. Additionally, areas for improvement were identified related to the visibility of clinical context, prediction explainability, and feedback management, highlighting the need to iterate again on the prototype before a potential real-world implementation.

Overall, the project demonstrates that the proposed solution addresses the main identified challenges and shows high potential as a product, while also highlighting the usefulness of the Design Thinking approach for developing complex solutions in the healthcare domain.

\vspace{.5cm}

\textbf{Keywords:} Design Thinking, genetic diagnosis, genomic variants, \texttt{FASTQ}, prototyping, artificial intelligence, user experience
