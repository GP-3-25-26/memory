\chapter{Conclusiones}\label{cap:conclusiones}

Podemos concluir que el proyecto ha sido un éxito y que la solución propuesta se encuentra bien encaminada, ya que el prototipo ha logrado cubrir los principales retos identificados en las fases iniciales del proceso. En especial, la automatización de la detección de variantes genómicas y el apoyo a la toma de decisiones clínicas han sido validados positivamente durante el testeo, demostrando un claro valor para los profesionales del ámbito genético.

Los resultados obtenidos confirman que el flujo de trabajo planteado es lógico, intuitivo y coherente con el modelo mental del usuario, y que la herramienta tiene un alto potencial para reducir de forma significativa el tiempo dedicado a tareas manuales y de investigación clínica. Esto refuerza la viabilidad del proyecto como base para el desarrollo de un producto real.

No obstante, el proceso de testeo también ha permitido identificar áreas de mejora que evidencian la naturaleza iterativa del Design Thinking. Aspectos como la visibilidad constante del contexto clínico, la mejora de la explicabilidad de las predicciones, la gestión del feedback y el autoguardado, así como pequeños bloqueos derivados del prototipado, ponen de manifiesto que sería recomendable volver a la fase de Prototipar para incorporar los cambios sugeridos por los usuarios.

De este modo, una nueva iteración del prototipo permitiría pulir la experiencia de uso, reforzar la confianza del usuario y mejorar la percepción de fiabilidad del sistema. Tras aplicar estas mejoras, resultaría coherente realizar un nuevo ciclo de testeo para validar que los ajustes realizados responden adecuadamente al feedback recibido.

En conjunto, el proyecto demuestra que se han abordado correctamente los problemas clave y que la solución propuesta aporta un valor real, confirmando que el proceso seguido ha sido adecuado. Al mismo tiempo, los aprendizajes obtenidos refuerzan la importancia de la iteración continua como elemento central para evolucionar la solución hacia un producto más maduro y alineado con las necesidades reales de los usuarios.
