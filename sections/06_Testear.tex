\chapter{Fase 5: Testear}\label{cap:testear}

En este capítulo se aborda la quinta y última fase de la metodología Design Thinking, la fase de \textbf{Testear}. El objetivo principal de esta etapa es \textbf{validar el prototipo} desarrollado en la fase anterior, obtener \textbf{retroalimentación} sobre su utilidad y usabilidad, y detectar \textbf{oportunidades de mejora} antes de plantear una iteración de diseño o desarrollo.

Para ello, se han seleccionado técnicas de validación y observación que permiten analizar cómo interactúan los usuarios con el prototipo, así como recopilar respuestas directas sobre el flujo y las funcionalidades. En los siguientes apartados se describen las técnicas seleccionadas y su justificación, y posteriormente se detalla cómo se aplicaron y qué resultados se obtuvieron.

\section{Técnicas utilizadas}

En la fase de Testear se realizó una selección de técnicas orientadas a la validación del prototipo y a la obtención de feedback. A continuación, se detallan las técnicas seleccionadas y aquellas descartadas, junto con la justificación correspondiente:

\begin{itemize}
    \item \textbf{Validación con usuarios reales.} Por desgracia no es posible utilizar genetistas reales para realizar la validación, pero sí que usaremos personas para que nos den su opinión sobre la utilidad y usabilidad de la aplicación.
    \item \textbf{Validación con usuarios mock.} Como hemos comentado antes, usaremos usuarios que no son genetistas para la validación.
    \item \textbf{Observación imparcial.} Sí la vamos a aplicar, no interferiremos entre los usuarios que prueban el prototipo.
    \item \textbf{Observación invisible.} La vamos a aplicar porque grabaremos las interacciones entre los usuarios y el prototipo en modo streaming para ver la reacción de los usuarios. Estas interacciones se analizarán después.
    \item \textbf{Preguntas de recapitulación.} Sí las vamos a aplicar. De este modo nos aseguramos de que el prototipo cumple lo que buscamos que solucione. Dentro de las preguntas de seguimiento existen algunas que dan pie a recibir sugerencias de mejora.
    \item \textbf{Juego de roles.} Se hará con herramientas de Inteligencia Artificial, lo comentaremos en detalle más adelante.
    \item \textbf{Juego de los 6 sombreros.} No la vamos a aplicar porque creemos que con todo lo comentado es más que suficiente.
    \item \textbf{Tarjetas rojas y verdes.} Tampoco la vamos a aplicar porque en la observación imparcial y la invisible extraemos la misma información sólo que de forma diferente.
    \item \textbf{Infografía.} La vamos a utilizar para sacar conclusiones sobre todo el feedback recibido y el resultado de la validación. La aplicación será idéntica a las veces anteriores que se usó.
\end{itemize}

\section{Aplicación de las técnicas}

A continuación vamos a comentar cómo se han aplicado las técnicas. Se contactó con una amiga de un miembro del grupo, estudiante de veterinaria, para consultar si podía colaborar en la validación. Para ello se le citó el \textbf{sábado 10 de enero de 2026 a las 17:45}. La sesión se realizó desde la comodidad del hogar mediante \textbf{Discord} \cite{discord}, y se planteó la posibilidad de grabar la pantalla con el objetivo de analizar posteriormente las interacciones del usuario con el prototipo.

Lamentablemente, la usuaria no accedió a grabar su rostro, por lo que no fue posible analizar gestos faciales por parte de todo el equipo. No obstante, uno de los miembros del grupo, en este caso Nicolás Herrera, sí estuvo presencialmente con ella para poder guiarla en caso de problemas técnicos y, además, para realizar la \textbf{observación imparcial}. A continuación se desgrana la experiencia, explicando cómo se cubrieron las técnicas y qué resultados se obtuvieron.

\subsection{Validación con usuarios mock}

Como se mencionó anteriormente, la persona que probó el prototipo no era genetista, sino una estudiante de último año de veterinaria. Aun así, posee conocimientos de genética y se le puso en contexto del problema previamente. Gracias a su colaboración, se obtuvo un feedback muy valioso. La aplicación de esta técnica consiste precisamente en dicha colaboración y en la recopilación de los resultados obtenidos.

Como evidencia, se proporciona el enlace al vídeo en el que se puede observar el testeo del prototipo junto con las preguntas de recapitulación:

\begin{center}
\url{https://youtu.be/48aC1PRzDik}
\end{center}

\subsection{Observación imparcial}

La aplicación de esta técnica consistió en observar las interacciones entre la usuaria y el prototipo, con la supervisión y análisis de Nicolás Herrera.

Para la observación imparcial se contó con la presencia de Nicolás, que estuvo con la usuaria mientras probaba el prototipo. Un detalle importante es que en ningún momento la usuaria escuchaba o interactuaba con los demás miembros del grupo: Nicolás era el único que podía escucharles. Aunque ambos estaban en el mismo espacio, la usuaria tuvo \textbf{control total} de la aplicación y se evitó ejercer presión sobre cómo debía usarla, con el objetivo de no generar incomodidad ni condicionar la experiencia. Como detalle adicional, se le comentó que tratara de decir en voz alta todo lo que se le pasara por la cabeza durante la interacción.

Se observó que en ciertas partes la usuaria parecía desconcertada, principalmente por alguna funcionalidad de la aplicación que esperaba que funcionara de otra manera, lo que provocaba que no supiera cómo avanzar y reflejaba en sus expresiones físicas que se encontraba bloqueada. Al sugerirle que probara el uso del chat, reflejó sorpresa, lo cual indica que en un principio no se dio cuenta de su presencia. Algo parecido ocurrió al acabar el análisis, ya que reflejó que no sabía qué hacer exactamente y fue necesario indicarle el botón donde se encontraba el inicio de un nuevo análisis.

En el resto del uso de la aplicación se notó un avance bastante cómodo, reflejando que le resultaba intuitiva y que su uso se convertía en una experiencia agradable.

\subsection{Observación invisible}

La aplicación de esta técnica consistió en tomar notas sobre las interacciones entre la usuaria y el prototipo en directo (por Discord), grabar la pantalla y realizar análisis posteriores.

Durante la validación se aplicó la observación invisible ya que Daniel, Fran y Jaime estuvieron escuchando, tomando notas y grabando todo para poder hacer un análisis exhaustivo mientras la usuaria probaba la aplicación. En el caso de Daniel, las notas se tomaron por WhatsApp ya que debió dejar el ordenador grabando la pantalla. La grabación se encuentra disponible en el siguiente enlace (desde el minuto 0:00 hasta el 03:40):

\begin{center}
\url{https://youtu.be/48aC1PRzDik}
\end{center}

A continuación pasamos a analizar las interacciones entre la usuaria y el prototipo.

Lo primero que se observó es que la usuaria presentó problemas para subir el archivo \texttt{.FASTQ} porque el prototipo no estaba pensado para que se subiera el archivo. Esto bloqueó temporalmente la experiencia y se tuvo que intervenir, ya que la usuaria dudó de si estaba utilizando bien el producto. Se considera que esto fue una mala práctica por parte del equipo. Afortunadamente, una vez comprendió lo ocurrido, la experiencia continuó sin más problemas.

Tras esto, la usuaria continuó usando la aplicación hasta llegar al listado de variantes. Al interactuar con el botón \textit{``Ver en ClinVar''}, notó que no funcionaba, dio por hecho que no era operativo y no lo volvió a utilizar. Ejecutó rápidamente los botones para ir pasando de fase, sin hacer uso del chat con la inteligencia artificial para resolver dudas, lo cual sugiere que no se percató de su presencia. Al terminar, hizo clic dos veces en \textit{``Confirmar y Guardar''}, lo que podría indicar que, tras esa acción, sería deseable que el sistema llevase directamente a la pestaña principal.

Al volver a usar la aplicación, intentó de nuevo arrastrar el archivo, sin éxito. Se le sugirió utilizar el chat con IA. En un principio no sabía qué preguntarle, lo que sugiere que se debería especificar mejor el propósito del chat para que quede más claro para los usuarios. Finalmente, probó a dejar un comentario del genetista, aunque la caja de comentarios no funcionó correctamente al introducir el texto. Tras volver al inicio, dio por finalizado el uso del prototipo.

\subsection{Preguntas de recapitulación}

Además de probar el prototipo, se realizaron preguntas una vez finalizó el uso con el objetivo de recopilar más información. Las preguntas habían sido preparadas previamente. A continuación se expone el listado de preguntas junto con sus respuestas obtenidas.

\begin{itemize}
    \item \textbf{¿El flujo te pareció lógico (subir FASTQ → procesar → ver variantes → predecir → comentar)?}
    \begin{itemize}
        \item Sí, me parece lógico.
    \end{itemize}

    \item \textbf{¿Crees que ha faltado algún paso durante el flujo de la aplicación?}
    \begin{itemize}
        \item Poder subir el documento correctamente, contar con el historial del paciente durante el proceso y poder hacer comentarios durante el proceso y no únicamente al final.
    \end{itemize}

    \item \textbf{¿Crees que ha sobrado algún paso durante el flujo de la aplicación?}
    \begin{itemize}
        \item No, todos los pasos son necesarios.
    \end{itemize}

    \item \textbf{¿Qué cambiarías del orden?}
    \begin{itemize}
        \item Lo que dijo en la pregunta anterior sobre añadir cosas al proceso, pero el orden en si lo ve lógico.
    \end{itemize}

    \item \textbf{En la pantalla de carga: ¿qué dato del paciente echaste en falta para poder “confiar” en el caso?}
    \begin{itemize}
        \item Que aparezca el historial (datos más concretos y no solo el nombre).
    \end{itemize}

    \item \textbf{En la pantalla de procesamiento: ¿qué te indica que el sistema “está trabajando bien”?}
    \begin{itemize}
        \item Aumento del porcentaje de carga para que se sepa al menos que está funcionando.
    \end{itemize}

    \item \textbf{En el listado de variantes: ¿qué información es imprescindible para decidir si una variante es relevante?}
    \begin{itemize}
        \item Las etiquetas que aparecen de colores.
    \end{itemize}

    \item \textbf{En la pantalla de predicción: ¿qué necesitarías ver para considerar la predicción útil en un entorno real?}
    \begin{itemize}
        \item Más información de la enfermedad (geografía, historial clínico del paciente, antecedentes familiares).
    \end{itemize}

    \item \textbf{En la pantalla de comentarios/feedback: ¿qué tipo de feedback te resultaría natural dar? (texto libre, etiquetas, “correcto/incorrecto”, etc.)}
    \begin{itemize}
        \item Confirmar y guardar lo ve interesante pero que trabajando cree que a alguien se le puede pasar guardarlo y que estaría bien que se guardase al menos como borrador para no perder la información en caso de que no se acuerde de darle a confirmar y guardar.
    \end{itemize}

    \item \textbf{¿La vinculación automática del 'Historial Clínico' del paciente con el análisis genómico te pareció intuitiva, o prefieres tener esos datos separados o presentados de otra forma antes de iniciar?"}
    \begin{itemize}
        \item Cree que debería de estar también presente.
    \end{itemize}

    \item \textbf{Si hoy tuvieras que usarlo en un caso real: ¿en qué te ayudaría más?}
    \begin{itemize}
        \item En comprobar la secuencia genética automáticamente y en determinar qué enfermedades son más probables, sus porcentajes y el por qué de esos porcentajes.
    \end{itemize}

    \item \textbf{¿Qué parte del proceso manual crees que más “recorta” esta herramienta?}
    \begin{itemize}
        \item Tiempo de investigar que tipo de enfermedad es la que coincide con el tipo de variación genética que tenga mi paciente, darle la probabilidad…
    \end{itemize}

    \item \textbf{¿Qué punto NO lo usarías (o te molestaría usarlo)?}
    \begin{itemize}
        \item En caso de que el historial del paciente sea muy claro de que puede ser una enfermedad u otra.
    \end{itemize}

    \item \textbf{¿Qué parte te quedó menos clara?}
    \begin{itemize}
        \item Considera que todo en la aplicación está claro
    \end{itemize}

    \item \textbf{¿Qué decisión clínica te ayudaría a tomar mejor (o más rápido)?}
    \begin{itemize}
        \item la usuaria expresa que el sistema ayuda a tomar decisiones clínicas en cuanto a diagnosticar si sus pacientes tienen predisposición a ciertas en enfermedades.
    \end{itemize}

    \item \textbf{Al revisar la lista de variantes detectadas, ¿el código de colores para 'Patogénica' vs 'Benigna' te ayudó a filtrar visualmente rápido, o sentiste que simplificaba demasiado la complejidad clínica?}
    \begin{itemize}
        \item La usuaria expresa que el impacto del código de colores es positivo para apreciar de manera rápida la diferencia entre variantes detectadas
    \end{itemize}

    \item \textbf{¿Qué error sería “inaceptable” que la herramienta cometiera?}
    \begin{itemize}
        \item la usuaria expresa que sería inaceptable que se equivocase en el porcentaje de probabilidad a la hora de dar una predicción.
    \end{itemize}

    \item \textbf{Preferirías: Menos predicciones pero más seguras, o más predicciones aunque alguna sea dudosa.}
    \begin{itemize}
        \item La usuaria expresa su predilección por una mayor cantidad de predicciones dudosas indicando el porcentaje de probabilidad sea menor ya que eso hace que el abanico de diagnóstico sea mucho más amplio, evitando obviar escenarios menos probables.
    \end{itemize}

    \item \textbf{Noté que [usaste/no usaste] el chat lateral. ¿Lo percibes como una herramienta de consulta rápida o como un asistente para redactar el informe final?}
    \begin{itemize}
        \item La usuaria percibió al asistente como una herramienta de consulta rápida.
    \end{itemize}

    \item \textbf{Si el chat pudiera realizar acciones por ti, ¿qué le pedirías? (Ej: 'Redacta un resumen para el paciente', 'Busca artículos recientes sobre este gen').}
    \begin{itemize}
        \item La usuaria expresa que pediría al asistente información sobre artículos relacionados a las enfermedades relacionadas con las predicciones.
    \end{itemize}

    \item \textbf{¿Para qué usarías el chat dentro de la app? (elige todas)}
    \begin{itemize}
        \item Preguntar por una variante concreta
        \item Preguntar por el significado clínico
        \item Pedir explicación de la predicción (seleccionado por la usuaria)
        \item Pedir siguientes pasos / pruebas recomendadas (seleccionado por la usuaria)
        \item No lo usaría
    \end{itemize}

    \item \textbf{¿Recomendarías esta herramienta a un profesional? ¿Por qué?}
    \begin{itemize}
        \item La usuaria sí recomendaría la herramienta a otros profesionales porque considera que ahorra mucho el tiempo y permite dar un diagnóstico a más pacientes de manera veloz.
    \end{itemize}

    \item \textbf{Si mañana desapareciera, ¿qué echarías de menos (si algo)?}
    \begin{itemize}
        \item La usuaria expresa que echaría de menos la rapidez con la que se ejecuta un posible diagnóstico, expresando la importancia de valorar el tiempo posterior dedicado a explorar dicho diagnóstico.
    \end{itemize}

    \item \textbf{¿Qué debería pasar para que esto sea “imprescindible” en vuestro trabajo?}
    \begin{itemize}
        \item La usuaria considera que para que eso ocurra, debería desaparecer el acceso a todas las demás fuentes de información, por ejemplo artículos científicos relacionados a la predicción de enfermedades genéticas, lo que obligaría a utilizar la aplicación por su velocidad.
    \end{itemize}
\end{itemize}

A continuación hemos realizado un análisis de las respuestas dadas por el usuario sobre:

\textbf{Puntos fuertes (validado)}
\begin{itemize}
    \item \textbf{Flujo lógico:} La secuencia "Carga -> Proceso -> Variantes -> Predicción" se ajusta al modelo mental del especialista.
    \item \textbf{Visualización de variantes:} El código de colores (Patogénica/Benigna) es efectivo y valorado para un filtrado rápido.
    \item \textbf{Valor principal:} Ahorro drástico de tiempo en la investigación bibliográfica y correlación genotipo-fenotipo.
    \item \textbf{Rol del chat:} Se percibe correctamente como una herramienta de apoyo/consulta rápida ("segunda pantalla") para buscar artículos y profundizar, sin reemplazar el criterio médico.
\end{itemize}

\textbf{Identificación de brechas y áreas de mejora}
\begin{itemize}
    \item \textbf{Contexto clínico persistente}
    \begin{itemize}
        \item \textbf{Problema:} El usuario echa en falta tener el historial y datos del paciente visibles durante todo el análisis ("Poder contar con el historial del paciente durante el proceso").
        \item \textbf{Solución propuesta:} Implementar una barra lateral o cabecera "sticky" con los datos clave del paciente (Fenotipo, Antecedentes) accesible en las pantallas de Variantes y Predicción.
    \end{itemize}

    \item \textbf{Fiabilidad y confianza}
    \begin{itemize}
        \item \textbf{Problema:} La "caja negra" genera desconfianza si faltan datos. Para confiar en la predicción, el usuario exige ver por qué (geografía, antecedentes familiares).
        \item \textbf{Solución propuesta:} En la pantalla de Explicabilidad (XAI), hacer explícito qué factores del historial clínico se usaron para el cálculo.
    \end{itemize}

    \item \textbf{Gestión del trabajo (borradores)}
    \begin{itemize}
        \item \textbf{Insight clave:} El miedo a perder el trabajo si se olvida dar a "Guardar".
        \item \textbf{Solución propuesta:} Implementar autoguardado o un estado de "Borrador" para los informes y comentarios.
    \end{itemize}

    \item \textbf{Comentarios en tiempo real}
    \begin{itemize}
        \item \textbf{Solicitud:} Poder hacer anotaciones durante el proceso, no solo al final.
        \item \textbf{Acción:} Añadir funcionalidad de notas rápidas o "flagging" en cada variante o etapa del análisis.
    \end{itemize}
\end{itemize}

\textbf{Perfil psicológico del usuario (insights)}
\begin{itemize}
    \item \textbf{Aversión al riesgo:} Prefiere una herramienta "pesimista" (más predicciones dudosas/bajas probabilidades) que una que oculte posibles diagnósticos. "Mejor que sobre a que falte".
    \item \textbf{Dependencia de la evidencia:} Solicita acceso a artículos científicos. La IA no es la autoridad final, es un bibliotecario acelerado.
    \item \textbf{Factor "imprescindible":} Considera que la herramienta se volvería indispensable si centralizara toda la información, eliminando la necesidad de buscar en fuentes externas manualmente.
\end{itemize}

\textbf{Conclusión de la validación}
\begin{itemize}
    \item La validación con el usuario especialista (genetista) arroja un resultado muy positivo en cuanto a la utilidad general y el buen planteamiento del flujo de trabajo. El usuario valora especialmente la velocidad y la capacidad de la herramienta para ahorrar tiempo de investigación, permitiendo diagnósticos más rápidos.
\end{itemize}

\textbf{Recomendaciones de iteración (next steps)}
\begin{itemize}
    \item \textbf{High priority:} Añadir el Resumen de Paciente (Mini-History) visible en la Pantalla 3 (Variantes) y 4 (Predicción).
    \item \textbf{High priority:} Modificar el algoritmo de predicción (simulado o real) y la UI para mostrar "Escenarios Alternativos" (enfermedades con menor probabilidad pero posibles), alineándose con su preferencia de seguridad.
    \item \textbf{Medium priority:} Integrar un botón "Buscar Artículos Relacionados" en el Chat o en la tarjeta de predicción.
    \item \textbf{Medium priority:} Implementar Auto-guardado de comentarios.
\end{itemize}

\textbf{Resumen final}
\begin{itemize}
    \item Existe una clara demanda de mayor contexto clínico (historial, antecedentes) visible durante todo el proceso, no solo al inicio. La confianza en la herramienta se basa en la transparencia (explicabilidad, porcentajes) y la precaución (preferencia por falsos positivos antes que falsos negativos).
\end{itemize}

\subsection{Juego de roles}

Para el juego de roles se ha propuesto usar la IA. Se acordó con el profesor para poder investigar nuevos usos de la inteligencia artificial al design thinking y también como investigación. Para la aplicación del juego de roles se va a utilizar el prototipo desarrollado, el IDE Antigravity y el LLM Gemini 3 Flash. La idea es usar la funcionalidad de dejar a la IA interactuar con el navegador para hacerle usar el prototipo y pedirle feedback. En este caso estaría siendo un usuario mock con el contexto que le pasemos con el prompt. Tras haber realizado el experimento, se comentarán y analizarán los resultados.

\textbf{Paso 1:} Se descargó Antigravity y se seleccionó el modelo. Tras esto, se diseñó un prompt, se configuró la extensión y se pegó el prompt en el chat. El prompt fue el siguiente:

\begin{quote}
«Analiza las fases 1 y dos del documento GP\_DesignThinking.pdf y cuando sepas cual es el problema imagina que eres un genetista. Una vez hecho eso, ve a Chrome, abre el navegador en http://localhost:5173/ y prueba el producto.
Cuando lo hayas probado, responde a las siguientes preguntas:
¿El flujo te pareció lógico (subir FASTQ → procesar → ver variantes → predecir → comentar)?
¿Crees que ha faltado algún paso durante el flujo de la aplicación?
¿Crees que ha sobrado algún paso durante el flujo de la aplicación?
¿Qué cambiarías del orden?
En la pantalla de carga: ¿qué dato del paciente echaste en falta para poder “confiar” en el caso?
En la pantalla de procesamiento: ¿qué te indica que el sistema “está trabajando bien”?
En el listado de variantes: ¿qué información es imprescindible para decidir si una variante es relevante?
En la pantalla de predicción: ¿qué necesitarías ver para considerar la predicción útil en un entorno real?
En la pantalla de comentarios/feedback: ¿qué tipo de feedback te resultaría natural dar? (texto libre, etiquetas, “correcto/incorrecto”, etc.)
¿La vinculación automática del 'Historial Clínico' del paciente con el análisis genómico te pareció intuitiva, o prefieres tener esos datos separados o presentados de otra forma antes de iniciar?"
Si hoy tuvieras que usarlo en un caso real: ¿en qué te ayudaría más?
¿Qué parte del proceso manual crees que más “recorta” esta herramienta?
¿En qué punto NO lo usarías (o te molestaría usarlo)?
¿Qué parte te quedó menos clara?
¿Qué decisión clínica te ayudaría a tomar mejor (o más rápido)?
Al revisar la lista de variantes detectadas, ¿el código de colores para 'Patogénica' vs 'Benigna' te ayudó a filtrar visualmente rápido, o sentiste que simplificaba demasiado la complejidad clínica?
¿Qué error sería “inaceptable” que la herramienta cometiera?
Preferirías: Menos predicciones pero más seguras, o más predicciones aunque alguna sea dudosa.
Noté que [usaste/no usaste] el chat lateral. ¿Lo percibes como una herramienta de consulta rápida o como un asistente para redactar el informe final?
Si el chat pudiera realizar acciones por ti, ¿qué le pedirías? (Ej: 'Redacta un resumen para el paciente', 'Busca artículos recientes sobre este gen').
¿Para qué usarías el chat dentro de la app? (elige una o varis)
Preguntar por una variante concreta
Preguntar por el significado clínico
Pedir explicación de la predicción
Pedir siguientes pasos / pruebas recomendadas
No lo usaría
¿Recomendarías esta herramienta a un profesional? ¿Por qué?
Si mañana desapareciera, ¿qué echarías de menos (si algo)?
¿Qué debería pasar para que esto sea “imprescindible” en vuestro trabajo?»
\end{quote}

\textbf{Paso 2:} Se abrió la carpeta del prototipo, se arrancó el proyecto y se introdujo el PDF en ella para que el LLM pudiera leerlo.

\begin{figure}[H]
    \centering
    \captionsetup{justification=centering}
    \includegraphics[width=1\textwidth]{figures/testear/juego_roles_paso2.png}
    \caption{Testear: Introducimos el PDF para el juego de roles.}
\end{figure}

\textbf{Paso 3:} Se abrió la extensión y se colocó la URL \url{http://localhost:5173/} en el navegador.

\textbf{Paso 4:} Se envió el prompt y se observó qué sucedía.

Se estaba grabando la pantalla, pero Gemini cortó la grabación porque realizó capturas de pantalla, lo que interrumpió el proceso de grabación. Para evitar perder toda la evidencia, se grabó parte de la sesión con un teléfono móvil. Lo poco que se consiguió documentar está accesible en YouTube. Aunque no se ve el inicio del proceso, sí se aprecia cómo responde e interactúa con el sistema por sí solo.

El vídeo está disponible en: \url{https://youtu.be/s24gMOg6xrk}

La respuesta de la IA fue:

\begin{quote}
«He completado el análisis de las fases 1 y 2 del documento de diseño y la evaluación del prototipo desde la perspectiva de un genetista clínico. He resumido mis hallazgos y las respuestas detalladas a tu cuestionario en el documento 
walkthrough.md

En resumen:

El flujo es lógico y ahorra una gran cantidad de trabajo manual de priorización y búsqueda bibliográfica.
La integración del historial clínico es el punto más fuerte para la confianza diagnóstica.
Sugiero añadir un paso de control de calidad (QC) inicial y mejorar la visibilidad de datos demográficos básicos (edad/sexo) en la carga.
El chat lateral se percibe como un asistente valioso para la redacción de informes y resolución de dudas rápidas sobre genes específicos.
Puedes revisar el proceso completo y las capturas de pantalla en el archivo adjunto.»
\end{quote}

El documento generado se adjuntó en el \nameref{feedback_gemini}.

\begin{figure}[H]
    \centering
    \captionsetup{justification=centering}
    \includegraphics[width=1\textwidth]{figures/testear/juego_roles_resultado.png}
    \caption{Testear: Documento generado por Gemini como resultado del juego de roles.}
\end{figure}

\subsection{Infografía}

Por último, se volvió a generar una infografía para cerrar la fase de testeo.

\begin{figure}[H]
    \centering
    \captionsetup{justification=centering}
    \includegraphics[width=1\textwidth]{figures/testear/infografia.png}
    \caption{Testear: Infografía.}
\end{figure}

