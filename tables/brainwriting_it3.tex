\renewcommand{\arraystretch}{1.15}
\setlength{\arrayrulewidth}{1.3pt}

\begin{landscape}
\centering

\begin{tabular}{|
    >{\centering\arraybackslash}m{2.2cm}|
    >{\arraybackslash}m{7cm}|
    >{\arraybackslash}m{7cm}|
    >{\arraybackslash}m{7cm}|
}
\hline
\rowcolor{lightgray}
\multicolumn{4}{|c|}{\textbf{Brainwriting (Iteración 3)}} \\
\hline
\rowcolor{lightgray}
\textbf{Miembro} & \textbf{Idea 1} & \textbf{Idea 2} & \textbf{Idea 3} \\
\hline

\cellcolor{lightgray!30}
\begin{center}
\textbf{Daniel}\\
{\footnotesize (revisado por Nicolás, Jaime)}
\end{center} &
Módulo integrado en Odoo que permite cargar \texttt{FASTQ} y un genoma de referencia para analizar automáticamente las diferencias genómicas y aplicar modelo que estime la probabilidad de desarrollar enfermedades genéticas. &
Modelo basado en LLM que, a partir de archivos \texttt{FASTQ} y del historial clínico del paciente, detecta variantes genómicas y predice posibles enfermedades, dando un porcentaje de probabilidad y una explicación comprensible. &
Aplicación que clasifica automáticamente variantes genéticas y ofrece un buscador avanzado para localizar variantes de interés y facilitar su análisis por parte del genetista. \\
\hline

\cellcolor{lightgray!30}
\begin{center}
\textbf{Francisco}\\
{\footnotesize (revisado por Daniel, Nicolás)}
\end{center} &
Procesamiento automático de \texttt{FASTQ} e historiales médicos mediante una aplicación web que permita la subida de archivos y genere un historial de variantes detectadas para su consulta futura. &
Consulta de diversas fuentes de datos y del historial clínico del paciente a través de una API GraphQL para informar al genetista sobre las variantes detectadas y su clasificación clínica. &
Entrenamiento de un modelo predictivo que utilice variantes seleccionadas por el genetista y datos clínicos del paciente, mostrando una predicción explicada en una interfaz HTML generada automáticamente. \\
\hline

\cellcolor{lightgray!30}
\begin{center}
\textbf{Jaime}\\
{\footnotesize (revisado por Francisco, Daniel)}
\end{center} &
Análisis automático de archivos \texttt{FASTQ} para detectar variantes genómicas sin intervención manual, permitiendo al genetista seleccionar las variantes de interés y descartar las benignas según el caso. &
Predicción de enfermedades mediante \textit{Machine Learning} a partir de variantes genómicas y bases de datos médicas, integrando información del historial clínico y ofreciendo un porcentaje de probabilidad. &
Integración del historial clínico y familiar del paciente en la predicción de enfermedades, con posibilidad de recomendar pruebas adicionales y ofrecer explicabilidad del resultado para su valoración clínica. \\
\hline

\cellcolor{lightgray!30}
\begin{center}
\textbf{Nicolás}\\
{\footnotesize (revisado por Jaime, Francisco)}
\end{center} &
Modelo de análisis y predicción que utiliza un listado de variantes seleccionadas por el genetista junto con el historial clínico para ofrecer una predicción con porcentaje de probabilidad y gráficas explicativas. &
Aplicación web que analiza diferencias genómicas y consulta APIs y bases de datos médicas para asociar variantes con enfermedades, incluyendo un sistema de feedback del genetista. &
Solución basada en dos scripts independientes para el análisis genético y la interpretación clínica, complementada con una interfaz web con gestión de permisos y seguridad de los datos del paciente. \\
\hline

\end{tabular}

\captionof{table}{Brainwriting (Iteración 3).}
\label{tab:brainwriting_iter3}
\end{landscape}
