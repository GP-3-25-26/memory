\renewcommand{\arraystretch}{1.15}
\setlength{\arrayrulewidth}{1.3pt}

\begin{landscape}
\centering

\begin{tabular}{|
    >{\centering\arraybackslash}m{2.2cm}|
    >{\arraybackslash}m{7cm}|
    >{\arraybackslash}m{7cm}|
    >{\arraybackslash}m{7cm}|
}
\hline
\rowcolor{lightgray}
\multicolumn{4}{|c|}{\textbf{Brainwriting (Iteración 4)}} \\
\hline
\rowcolor{lightgray}
\textbf{Miembro} & \textbf{Idea 1} & \textbf{Idea 2} & \textbf{Idea 3} \\
\hline

\cellcolor{lightgray!30}
\begin{center}
\textbf{Daniel}\\
{\footnotesize (revisado por Nicolás, Jaime, Francisco)}
\end{center} &
Módulo integrado en Odoo que permite analizar \texttt{FASTQ} y el historial clínico del paciente para predecir la probabilidad de desarrollar enfermedades genéticas como apoyo a la decisión del genetista. &
Modelo basado en LLM y agentes autónomos que, a partir de variantes y del historial clínico, predice enfermedades proporcionando un porcentaje de probabilidad y una explicación del resultado. &
Aplicación web que clasifica automáticamente variantes genéticas y ofrece un buscador avanzado para localizarlas según el historial clínico del paciente. \\
\hline

\cellcolor{lightgray!30}
\begin{center}
\textbf{Francisco}\\
{\footnotesize (revisado por Daniel, Nicolás, Jaime)}
\end{center} &
Aplicación web para la subida y procesamiento automático de archivos \texttt{FASTQ} e historiales médicos, detectando variantes genéticas y almacenándolas en un historial para su trazabilidad y reutilización. &
Servicio basado en una API GraphQL que consulta de forma unificada diversas fuentes de datos médicas y el historial clínico del paciente para clasificar variantes y asociarlas con patologías relevantes. &
Entrenamiento de un modelo predictivo que utiliza variantes seleccionadas por el genetista y datos clínicos del paciente, mostrando una predicción explicada en una interfaz visual basada en un archivo HTML. \\
\hline

\cellcolor{lightgray!30}
\begin{center}
\textbf{Jaime}\\
{\footnotesize (revisado por Francisco, Daniel, Nicolás)}
\end{center} &
Análisis automático de \texttt{FASTQ} para detectar variantes sin intervención manual, permitiendo la selección manual de variantes de interés y la búsqueda opcional de enfermedades asociadas en historiales médicos. &
Predicción enfermedades hereditarias y no hereditarias usando \textit{Machine Learning}, integrando variantes y datos clínicos del paciente para ofrecer un porcentaje de probabilidad basado en el aprendizaje del modelo. &
Integración historial clínico y familiar en la predicción de enfermedades, aportando información sobre la posible raíz de la enfermedad, recomendando pruebas adicionales y ofreciendo explicabilidad. \\
\hline

\cellcolor{lightgray!30}
\begin{center}
\textbf{Nicolás}\\
{\footnotesize (revisado por Jaime, Francisco, Daniel)}
\end{center} &
Modelo de análisis y predicción que utiliza variantes genéticas seleccionadas por el genetista y el historial clínico del paciente para ofrecer una predicción con porcentaje de probabilidad, gráficas explicativas y sugerencias de pruebas médicas. &
Aplicación web que analiza diferencias genómicas, consulta APIs y bases de datos médicas y ofrece un resultado estructurado con sistema de feedback para el genetista, incluyendo la detección de variantes no registradas previamente. &
Solución basada en dos scripts independientes para el análisis genético y la interpretación clínica, complementada con una interfaz web segura con gestión de permisos, donde los pacientes pueden consultar los resultados. \\
\hline

\end{tabular}

\captionof{table}{Brainwriting (Iteración 4).}
\label{tab:brainwriting_iter4}
\end{landscape}
