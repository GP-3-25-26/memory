\renewcommand{\arraystretch}{1.15}
\setlength{\arrayrulewidth}{1.3pt}

\begin{landscape}
\centering

\begin{tabular}{|
    >{\centering\arraybackslash}m{2.2cm}|
    >{\arraybackslash}m{7cm}|
    >{\arraybackslash}m{7cm}|
    >{\arraybackslash}m{7cm}|
}
\hline
\rowcolor{lightgray}
\multicolumn{4}{|c|}{\textbf{Brainwriting (Iteración 1)}} \\
\hline
\rowcolor{lightgray}
\textbf{Miembro} & \textbf{Idea 1} & \textbf{Idea 2} & \textbf{Idea 3} \\
\hline

\cellcolor{lightgray!30}\textbf{Daniel} &
Módulo de Odoo que interactúe con una API que prediga enfermedades genéticas. &
LLM que, dado un archivo \texttt{FASTQ} y un historial clínico, detecte variantes y prediga enfermedades genéticas. &
Aplicación para clasificar variantes genéticas. \\
\hline

\cellcolor{lightgray!30}\textbf{Francisco} &
Procesamiento automático de archivos \texttt{FASTQ} para encontrar variantes genéticas mediante una aplicación web que permita la subida de dichos archivos. &
Consulta de diversas fuentes de datos a través de APIs para informar al genetista sobre las variantes detectadas. &
Entrenamiento de un modelo predictivo que utilice variantes seleccionadas por el genetista y datos clínicos del paciente, mostrando una predicción explicada en la interfaz de usuario. \\
\hline

\cellcolor{lightgray!30}\textbf{Jaime} &
Análisis automático de archivos \texttt{FASTQ}: sistema que analiza automáticamente archivos \texttt{FASTQ} y detecta variantes genómicas sin intervención manual. &
Predicción de enfermedades con \textit{Machine Learning}: modelo que predice enfermedades hereditarias a partir de variantes genómicas y bases de datos médicas. &
Integración de historial clínico y familiar: módulo que incorpora el historial clínico y familiar del paciente en la predicción de enfermedades. \\
\hline

\cellcolor{lightgray!30}\textbf{Nicolás} &
Aportar un modelo que, tras analizar archivos genómicos, devuelva el nombre de la enfermedad, un porcentaje de probabilidad de acierto y gráficas explicativas que ayuden al genetista a interpretar el resultado. &
Aplicación web que analiza automáticamente las principales diferencias entre archivos genómicos y, mediante llamadas a distintas APIs, identifica la enfermedad asociada. &
Solución compuesta por dos scripts: uno encargado de analizar los archivos y obtener las diferencias, y otro que consulta distintas fuentes para identificar la enfermedad asociada. \\
\hline

\end{tabular}

\captionof{table}{Brainwriting (Iteración 1).}
\label{tab:brainwriting_iter1}
\end{landscape}
